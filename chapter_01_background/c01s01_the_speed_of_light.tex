The idea that light travels at a finite speed is very old,
the writings of the Greek philosopher Empedocles (\textit{c}. 492 -- 430 B.C.E)
pre-date Euclidean geometry by at least 100 years
[\cite[p.~248]{SartonAncientScienceThroughTheGoldenAgeOfGreece}].
Empedocles believed that light was produced by the eye and then propagated
outwards at a finite speed. Euclid himself adopted this belief, writing
[\cite[p.~1]{BurtonTheOpticsOfEuclid}]:

\begin{quotation}
    Let it be assumed that lines drawn pass from the eye
    through a space of great extent.

    Euclid of Alexandria,
    Optics, c. 300 B.C.E.
\end{quotation}

Euclid used this as the very first postulate in his physical model for light.
\footnote{
    Euclid's book is not without value, even in modern times.
    It discusses the mathematics behind geometric optics, and accurately
    describes the laws of reflection.
}
The Roman astronomer Ptolemy (\textit{c}. 100 -- 170 C.E.) also repeated this
idea in his book \textit{Optics}
[\cite[p.~22-23]{SmithPtolemysTheoryOfVisualPerception}].
\footnote{
    The original Greek text has long been lost.
    The 1996 English translation is based on Albert Lejeune's edits made in
    1956 to the original Latin translation, which were published 1154 by
    the Sicilian admiral Eugenius of Palermo (\textit{c}. 1130 -- 1202 C.E.).
    This Latin translation was itself a translation of the Arabic translation.
    Like the Greek original, the Arabic translation has also been lost to time,
    but it is generally believed that it was based off of the original Greek
    writings.
}

While many prominent astronomers and mathematicians supported the idea that
light travels at a finite speed, the Greek philosopher Aristotle
(384 - 322 B.C.E.) found a major flaw in it. If light originates from the eye,
and if light travels at a finite speed, then how would we be able to see things
that are very far away the moment we open our eyes
[\cite[p.~255]{%
    MacKayOldfordScientificMethodStatisticalMethodAndTheSpeedOfLight%
}]?
The Greek mathematician Hero of Alexandria
(\textit{c}. 100 -- \textit{c}. 200 C.E.)
\footnote{
    Famous for Heron's method of root finding, and Hero's principle of least
    time, which accurately describes the \textit{refraction} of light as it
    passes from one medium to another. In modern physics textbooks this is
    usually referred to as \textit{Fermat's} principle, named after the French
    mathematician and physicist Pierre de Fermat (1601 -- 1665 C.E.).
}
considered this observeration and further noted that since the stars are so very
far away, for humans to observe them would require the speed of light to be
infinite.
\footnote{
    Hero of Alexandria did not have estimates for how far away the stars were
    to him. This first came with the German astronomer and mathematician
    Friedrich Bessel (\textit{c}. 1784 -- 1846) in 1838.
    \textit{Bessel functions}, which were discovered by Bessel, will be used
    throughout this book.
}

For nearly two thousand years the Aristotle belief was held by most scientists
and philosophers, and there was no interest in trying to measure the speed of
light. In 1638 the Italian astronomer Galileo Galilei
(1564 -- 1642 C.E.) proposed an experiment in which two observers standing
atop mountains that are around two kilometers apart would
open a lantern and try to determine the time delay between
uncovering the lantern and seeing the light
[\cite[p.~24-25]{BoyerEarlyEstimatesOfTheVelocityOfLight},
\cite[p.~1252]{FoschiLeoneGalileoMeasurementOfTheVelocityOfLight},
\cite[p.~50]{GalileoTwoNewSciences}].
\footnote{
    Galileo's original work is in Italian and Latin. There is an English
    translation by Stillman Drake published by Wisconsin University Press
    (1974).
}
Galileo admits he was unable to determine whether or not
light travels at a finite speed from these experiments,
but concedes that if it is finite, it must travel at
several orders of magnitude more than the speed of sound.

Galileo's experiment was doomed to fail. The speed of
light is so great that by the time it travels from
the lantern to the observer, the electrical signals
in the human nervous system have not yet traveled from the
observer's eye to their brain. The modern measurement of the
speed of light is 299,792,458 meters per second
[\cite{CGPMDefinitionOfTheMeter}].
\footnote{
    This value is exact, by definition. The length of a meter is defined to be
    precisely the distance light travels through vacuum in
    1/299,792,458 seconds.
}
The fastest neurons in the human body send electochemical signals at
120 meters per second. In the time it takes light to travel
the five kilometers between the mountains the observers stood
on, these signals have only traveled two millimeters.
Galileo's conclusion that light must be orders of magnitude
greater than that of sound is correct: the ratio is nearly $10^{6}$.

While Galileo's methods failed to accurately determine the speed
of light, only a few decades later were physicists able to make
such measurements. The key lies in the planet Jupiter and its many
moons. In Isaac Newton's 1704 text \textit{Optiks} he writes:

\begin{quotation}
    Mathematicians usually consider the Rays of Light to be Lines
    reaching from the luminous Body to the Body illuminated, and the
    refraction of those Rays to be the bending or breaking of those
    lines in their passing out of one Medium into another. And thus
    may Rays and Refractions be considered, if Light be propagated in
    an instant. But by an Argument taken from the Æquations of the times
    of the Eclipses of Jupiter’s Satellites, it seems that Light is
    propagated in time, spending in its passage from the Sun to us
    about seven Minutes of time: And therefore I have chosen to
    define Rays and Refractions in such general terms as may agree
    to Light in both cases.

    Isaac Newton,
    Optiks, 1704
\end{quotation}

Newton is eluding to an argument made by Ole R{\o}mer.
R{\o}mer observes that, if Newton's theory of gravity is true,
then to explain disrcepencies in the period of Io around Jupiter, one
must conclude that the speed of light is finite. Using estimates of
the distance from the Earth to the Sun, and by measuring the difference
in the period to be about 22 minutes, we obtain a value of
220,000,000 meters per second. Not bad!

Better measurements did not take long.
In 1729 James Bradley used the concept of \textit{stellar abberation}
to obtain a measurement that has an error of less than 2\%
from the modern value.

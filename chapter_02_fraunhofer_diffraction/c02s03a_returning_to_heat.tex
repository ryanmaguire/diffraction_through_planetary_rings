Let us briefly return to heat. The solution is:

\begin{equation}
    u(x,\,t)
    =
    \sum_{n=1}^{\infty}a_{n}
        \sin\left(\frac{n\pi{x}}{L}\right)
        \exp\left(-\frac{n^{2}\pi^{2}\alpha{t}}{L^{2}}\right)
\end{equation}

Given the \textit{boundary condition} $u(x,\,0)=f(x)$, this reduces to:

\begin{equation}
    f(x)
    =
    \sum_{n=1}^{\infty}a_{n}
        \sin\left(\frac{n\pi{x}}{L}\right)
\end{equation}

This is called a \textbf{Fourier series}. In particular, this is the Fourier
expansion for the function $f$. How do we determine what the coefficients are?
We first note that the sine function satisfies a certain
\textit{orthogonality condition}. That is, for integers $m$ and $n$, we have:

\begin{equation}
    \frac{2}{L}\int_{0}^{L}
        \sin\left(\frac{n\pi{x}}{L}\right)
        \sin\left(\frac{m\pi{x}}{L}\right)\,
        \textrm{d}x
    =
    \begin{cases}
        1,&m=n\\
        0,&m\ne{n}
    \end{cases}
\end{equation}

The right-hand side is common enough to warrant a notation.
The \textit{Kronecker-$\delta$} function is defined via:

\begin{equation}
    \delta_{m,\,n}
    =
    \begin{cases}
        1,&m=n\\
        0,&m\ne{n}
    \end{cases}
\end{equation}

We can use this property to develop a formula for the coefficients
of the Fourier expansion. First, we assume that $f$ is a \textit{nice}
function, meaning we can swap sums and integrals at will. We write:

\begin{equation}
    \begin{array}{lrcl}
        &
        \displaystyle
        f(x)\sin\left(\frac{m\pi{x}}{L}\right)
        &=&
        \displaystyle
        \sum_{n=1}^{\infty}
        a_{n}\sin\left(\frac{n\pi{x}}{L}\right)
            \sin\left(\frac{m\pi{x}}{L}\right)\\
        \Longrightarrow
        &
        \displaystyle
        \int_{0}^{L}f(x)
            \sin\left(\frac{m\pi{x}}{L}\right)\,
            \textrm{d}x
        &=&
        \displaystyle
        \int_{0}^{L}
            \sum_{n=1}^{\infty}
                a_{n}\sin\left(\frac{n\pi{x}}{L}\right)
                \sin\left(\frac{m\pi{x}}{L}\right)\,
                \textrm{d}x\\
        \Longrightarrow
        &
        \displaystyle
        \int_{0}^{L}f(x)
            \sin\left(\frac{m\pi{x}}{L}\right)\,
            \textrm{d}x
        &=&
        \displaystyle
        \sum_{n=1}^{\infty}
            a_{n}
            \int_{0}^{L}
                \sin\left(\frac{n\pi{x}}{L}\right)
                \sin\left(\frac{m\pi{x}}{L}\right)\,
                \textrm{d}x\\
        \Longrightarrow
        &
        \displaystyle
        \int_{0}^{L}f(x)
            \sin\left(\frac{m\pi{x}}{L}\right)\,
            \textrm{d}x
        &=&
        \displaystyle
        \sum_{n=1}^{\infty}a_{n}\frac{L\delta_{m,\,n}}{2}\\
        \Longrightarrow
        &
        \displaystyle
        a_{m}
        &=&
        \displaystyle
        \frac{2}{L}
        \int_{0}^{L}f(x)
            \sin\left(\frac{m\pi{x}}{L}\right)\,
            \textrm{d}x
    \end{array}
\end{equation}

So, does this work? Let's try an example.
Consider $f:[0,\,1]\rightarrow\mathbb{R}$ give by:

\begin{equation}
    f(x)
    =
    x(1-x)
\end{equation}

Simple enough, and it satisfies $f(0)=0$ and $f(1)=0$.
The Fourier coefficients are then:

\begin{equation}
    \begin{array}{rcl}
        \displaystyle
        a_{n}
        &=&
        \displaystyle
        2\int_{0}^{1}x(1-x)\sin(n\pi{x})\,
            \textrm{d}x\\
        \displaystyle
        &=&
        \displaystyle
        \frac{4}{\pi^{3}}
        \frac{1-(-1)^{n}}{n^{3}}
    \end{array}
\end{equation}

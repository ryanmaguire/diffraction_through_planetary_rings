Returning to the normal heat equation, we now have a means of
solving it using \textbf{Fourier series}, and can compute the
coefficients using the integral formula:

\begin{equation}
    a_{n}
    =
    \frac{2}{L}
    \int_{0}^{L}f(x)\sin\left(
        \frac{n\pi{x}}{L}
    \right)\,\textrm{d}x
\end{equation}

There are issues about convergence, swapping integrals and sums, etc.,
but such problems reside in Fourier analysis. Surprising as it may be,
the real reason this works is that the sine function satisfies the
following differential equation.

\begin{equation}
    y^{\prime\prime}(x)
    +
    y(x)=0
\end{equation}

We can consider a more general differential equation,
called the \textbf{Sturm-Liouvile Equation}:

\begin{equation}
    \frac{\textrm{d}}{\textrm{d}x}
    \left(
        \alpha(x)
        \frac{\textrm{d}y}{\textrm{d}x}
    \right)
    +\big(
        \lambda{w}(x)+\beta(x)
    \big)y(x)
    =0
\end{equation}

where $\alpha$, $\beta$, and $w$ are smooth functions
defined on an interval $[a,\,b]$, and where $\alpha$ and
$w$ are \textit{positive} on this interval as well.
Given such a problem, there are positive real numbers
$0<\lambda_{0}<\lambda_{1}<\cdots$ called \textit{eigenvalues}
and corresponding functions $\varphi_{n}$ called
\textit{eigenfunctions} satisfying the
differential equation, and such that for any
piece-wise continuous
(not necessarily differentiable, or even continuous!)
function $f:[a,\,b]\rightarrow\mathbb{R}$, $f$ can be
written as a sum:

\begin{equation}
    f(x)
    =
    \sum_{n=0}^{\infty}
        a_{n}\varphi_{n}(x)
\end{equation}

and the coefficients are given by:

\begin{equation}
    a_{n}
    =
    \int_{a}^{b}
        f(x)\varphi_{n}(x)w(x)\,
        \textrm{d}x
\end{equation}

and moreover, the eigenfunctions are \textit{orthogonal}
with respect to $w$:

\begin{equation}
    \int_{a}^{b}
        \varphi_{m}(x)\varphi_{n}(x)w(x)\,
        \textrm{d}x
    =
    \delta_{m,\,n}
\end{equation}

Fourier series arise as a special case of Sturm-Liouville
theory by setting $\alpha(x)=w(x)=1$ and $\beta(x)=0$.
We could also consider the following differential equation:

\begin{equation}
    x^{2}y^{\prime\prime}(x)
    +
    xy^{\prime}(x)
    +
    (x^{2}-\lambda^{2})y(x)
    =
    0
\end{equation}

This is the \textbf{Bessel equation} and the solutions are
used to model acoustics, heat, and \textit{electromagnetic waves}
(we'll return to these soon). The solution is given by a
power series:

\begin{equation}
    J_{n}(x)
    =
    \sum_{k=0}^{\infty}
        \frac{(-1)^{k}}{k!(n+k)!}\left(
            \frac{x}{2}
        \right)^{2k+n}
\end{equation}

The rest of the Bessel functions can be computed
using a \textit{recursion} relation. The Bessel functions satisfy
the following:

\begin{equation}
    J_{n+2}(x)
    =
    \frac{2(n+1)}{x}J_{n+1}(x)
    -
    J_{n}(x)
\end{equation}

Another useful relation relates the derivatives
of the Bessel functions:

\begin{equation}
    \frac{\textrm{d}}{\textrm{d}x}
    \left(
        x^{n+1}J_{n+1}(x)
    \right)
    =
    x^{n+1}J_{n}(x)
\end{equation}

Lastly, we will make use of the Bessel integral formula:

\begin{equation}
    J_{0}(x)
    =
    \frac{1}{2\pi}
    \int_{0}^{2\pi}
        \exp\big(ix\cos(t)\big)\,
        \textrm{d}t
\end{equation}

Our first application of all of this is in deriving a formula
for the intensity of light through a cylindrically symmetric
aperture in the Fraunhofer limit. In the case of a circle, we
obtain \textit{Airy disks}.

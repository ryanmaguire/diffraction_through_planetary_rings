The diffraction pattern for a straight edge is given by:

\begin{equation}
    \def\arraystretch{3em}
    \begin{array}{rcl}
        \displaystyle
        \hat{T}(\rho_{0})
        \!\!
        &=&
        \!\!
        \displaystyle
        \frac{1+i}{2F}
        \int_{-\infty}^{\infty}
            T(\rho)
            \exp\left(
                i\frac{\pi}{2}\left(
                    \frac{\rho-\rho_{0}}{F}
                \right)^{2}
            \right)\,\textrm{d}\rho\\
        \!\!
        &=&
        \!\!
        \displaystyle
        \frac{1+i}{2F}
        \int_{R}^{\infty}
            \exp\left(
                i\frac{\pi}{2}
                \left(
                    \frac{\rho-\rho_{0}}{F}
                \right)^{2}
            \right)\,\textrm{d}\rho
    \end{array}
\end{equation}

where $R>0$ is a positive real number, the radius where the straight
edge occurs. Using Euler's formula, this becomes:

\begin{equation}
    \hat{T}(\rho_{0})
    =
    \frac{1+i}{2F}\int_{R}^{\infty}\left(
        \cos\left(\frac{\pi}{2}\left(\frac{\rho-\rho_{0}}{F}\right)^{2}\right)
        +i\sin\left(\frac{\pi}{2}\left(\frac{\rho-\rho_{0}}{F}\right)^{2}\right)
    \right)\,\textrm{d}\rho
\end{equation}

The **normalized Fresnel integrals** are defined via:

\begin{equation}
    \def\arraystretch{2.5em}
    \begin{array}{rcl}
        \displaystyle
        C(x)
        \!\!
        &=&
        \!\!
        \displaystyle
        \int_{0}^{x}
            \cos\left(\frac{\pi}{2}t^{2}\right)\,\textrm{d}t\\
        S(x)
        \!\!
        &=&
        \!\!
        \displaystyle
        \int_{0}^{x}
            \sin\left(\frac{\pi}{2}t^{2}\right)\,\textrm{d}t
    \end{array}
\end{equation}

These are provided by \texttt{tmpyl}, see the example above.
Use these to write out explicit formulas for the diffraction pattern
and plot them using Python.
